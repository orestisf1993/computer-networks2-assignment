\chapter{Εισαγωγή}
Η παρούσα εργασία έγινε στα πλαίσια του μαθήματος
\href{http://qa.auth.gr/el/class/1/600010175}{"Δίκτυα Υπολογιστών ΙΙ"},
του τμήματος \href{http://ee.auth.gr/}{Ηλεκτρολόγων Μηχανικών \& Μηχανικών Ηλεκτρονικών Υπολογιστών}
του \href{https://www.auth.gr/}{Αριστοτελείου Πανεπιστημίου Θεσσαλονίκης}.
Αφορά το socket programming υλοποιημένο μέσω της γλώσσας προγραμματισμού Java.
Επιβλέπων και διδάσκων καθηγητής είναι ο
\href{http://ithaki.eng.auth.gr/activities/cv-gr.html}{κ. Δημήτριος Μητράκος}.

\section{Περιγραφή της εφαρμογής}
Στόχος της εφαρμογής είναι η σύνδεση και επικοινωνία με τον server
\href{http://ithaki.eng.auth.gr/netlab/index.html}{ithaki}
και στόχοι της αποτελούν:
\begin{itemize}
\item Την εξοικείωση με τα πρωτόκολλα επικοινωνίας υπολογιστών \href{https://en.wikipedia.org/wiki/User_Datagram_Protocol}{UDP (User Datagram Protocol)} και
\href{https://en.wikipedia.org/wiki/Transmission_Control_Protocol}{TCP (Transmission Control Protocol)}.
\item Την εισαγωγή στους μηχανισμούς μετάδοσης ψηφιακού ήχου σε πραγματικό χρόνο διαμέσου δικτύων μεταγωγής πακέτων
(\href{https://en.wikipedia.org/wiki/Differential_pulse-code_modulation}{DPCM} \&
\href{https://en.wikipedia.org/wiki/Adaptive_differential_pulse-code_modulation}{AQ-DPCM}).
\item Τη συλλογή στατιστικών μετρήσεων παραμέτρων που σχετίζονται με την ποιότητα της επικοινωνίας των υπολογιστών στο Internet.
\end{itemize}
