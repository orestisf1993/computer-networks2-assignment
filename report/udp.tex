\section{Το πρωτόκολλο UDP}
Το πρωτόκολλο UDP (User Datagram Protocol ή Universal Datagram Protocol)
είναι ένα από τα βασικά πρωτόκολλα που χρησιμοποιούνται στο Internet.
Σχεδιάστηκε το 1980 και ορίζεται στο RFC 768\cite{rfc768}\cite{wiki:udp}.

Αποτελεί ένα απλό,
\href{https://en.wikipedia.org/wiki/Connectionless_communication}{connectionless}
πρωτόκολλο στο οποίο δεν υπάρχουν διάλογοι
\href{https://en.wikipedia.org/wiki/Handshaking}{"handshaking"}.
Δηλαδή, η επικοινωνία μέσω του UDP δεν προσφέρει την δυνατότητα στις συνδεδεμένες συσκευές να "θυμούνται" που βρίσκονται σε μια "συνομιλία" ανταλλαγής μηνυμάτων.
Η επικοινωνία γίνεται μέσω σύντομων μηνυμάτων (γνωστών και ως
\href{https://en.wikipedia.org/wiki/Datagram}{datagrams}
) από τον έναν υπολογιστή στον άλλον μέσα σε ένα δίκτυο υπολογιστών.

Ένα από τα κύρια χαρακτηριστικά του UDP είναι ότι δεν εγγυάται αξιόπιστη επικοινωνία.
Τα πακέτα UDP που αποστέλλονται από έναν υπολογιστή μπορεί να φτάσουν στον παραλήπτη με λάθος σειρά, διπλά ή να μην φτάσουν καθόλου ανάλογα με την τρέχουσα κατάσταση του δικτύου.
Αντιθέτως, το πρωτόκολλο
\href{https://en.wikipedia.org/wiki/Transmission_Control_Protocol}{TCP}
διαθέτει όλους τους απαραίτητους μηχανισμούς ελέγχου και επιβολής της αξιοπιστίας και μπορεί να εγγυηθεί την αξιόπιστη επικοινωνία μεταξύ των υπολογιστών.
Ωστόσο, λόγω της έλλειψης των μηχανισμών αυτών, το πρωτόκολλο UDP καθιστάται αρκετά πιο γρήγορο και αποτελεσματικό

\subsection{Χρήσεις του UDP}
%TODO: own book citations.
Βάση των χαρακτηριστικών του, το UDP σήμερα χρησιμοποιείται κυρίως ως εξής:
\begin{itemize}
\item Σε εφαρμογές audio \& video streaming.
Για τις εφαρμογές αυτές είναι πολύ σημαντικό τα πακέτα να παραδοθούν στον παραλήπτη σε σύντομο χρονικό διάστημα, ούτως ώστε να μην υπάρχει διακοπή στην ροή του ήχου ή της εικόνας.

Στο TCP η διόρθωση λαθών απαιτεί αναμετάδοση των πακέτων που προσθέτει καθυστερήσεις και αυξημένο bandwith.
Το streaming απαιτεί ένα πρωτόκολλο που μπορεί να αγνοεί λάθη στα δεδομένα
\nocite{video-audio-streaming}
\cite{video-audio-streaming-15-17}.

Δεδομένου ότι το UDP δεν εγγυάται την παράδοση των πακέτων,
ο δέκτης πρέπει να βασίζεται σε ανώτερα επίπεδα (πχ RTP) για να ανιχνεύσει απώλεια πακέτων
\cite{911156}\cite{wiki:rtp}.

\item \href{https://en.wikipedia.org/wiki/Domain_Name_System}{Domain Name System (DNS)}.

\item \href{https://en.wikipedia.org/wiki/Dynamic_Host_Configuration_Protocol}{Dynamic Host Configuration Protocol (DHCP)}.

\item Διαδικτυακά παιχνίδια (MMORPGs).\cite{games}

\item \href{https://en.wikipedia.org/wiki/Voip}{Voice over IP (VoIP)}.
\end{itemize}

\subsection{Σύγκριση UDP και TCP}
Αναφέρονται συνοπτικά οι πιο βασικές διαφορές μεταξύ των πρωτοκόλλων UDP και TCP\cite{wiki:udp}\cite{wiki:tcp}\cite{udp-technet}:
\begin{table}[H]
\begin{tabularx}{\linewidth}{|X|X|}
\hline
\multicolumn{1}{|c|}{\bfseries UDP} & \multicolumn{1}{c|}{\bfseries TCP}\\\hline
Αναξιόπιστο. Τα προγράμματα που χρησιμοποιούν UDP πρέπει να αναλάβουν την αντιμετώπιση σφαλμάτων και χαμένων δεδομένων ανεξάρτητα.&
Αξιόπιστο. Χρησιμοποιεί διάφορους μηχανισμούς ούτως ώστε να διασφαλιστεί ότι τα πακέτα που μεταδίδονται από τον αποστολέα θα φτάσουν σίγουρα στον παραλήπτη.
Χρησιμοποιείται επιβεβαίωση λήψης πακέτου από τον παραλήπτη και επαναποστολή πακέτων που χάθηκαν.\\\hline

Τα μηνύματα φτάνουν στο παραλήπτη με σειρά που δεν μπορεί να προβλεφθεί.&
Τα μηνύματα φτάνουν στο παραλήπτη με τη σειρά που στάλθηκαν.\\\hline

Δεν υπάρχει συνεδρία μεταξύ δέκτη και παραλήπτη.&
Υπάρχει συνεδρία.\\\hline

Ελαφρύ (low overhead). Δεν χρειάζονται επιπλέον πακέτα πριν την αρχή μετάδοσης πακέτων δεδομένων.&
Βαρύ. Χρειάζονται τουλάχιστον 3 πακέτα για την εγκαθίδρυση της σύνδεσης, πριν ακόμη μεταδοθεί οποιοδήποτε πακέτο δεδομένων. Επίσης, οι μηχανισμοί αξιοπιστίας που υλοποιεί το κάνουν ακόμη πιο βαρύ.\\\hline

Δυνατότητα \href{https://en.wikipedia.org/wiki/Broadcasting}{broadcasting}. Λόγω της έλλειψης συνεδρίας, τα πακέτα από τον αποστολέα μπορούν να σταλούν σε οποιαδήποτε συσκευή.&
Δεν υπάρχει τέτοια δυνατότητα.\\\hline

Χρήση datagram. Κάθε πακέτο UDP ονομάζεται επίσης και datagram.
Θεωρείται μια μεμονωμένη οντότητα που θα πρέπει να μεταδοθεί ολόκληρη.
Κατά συνέπεια, δεν υποστηρίζεται η διοχέτευση πακέτων μέσα σε ένα κανάλι.&
Streaming. Τα δεδομένα θεωρούνται ένα stream από bytes. Δεν υπάρχει κάποια λειτουργία για τον διαχωρισμό τον μηνυμάτων σύμφωνα με τα όριά τους.\\\hline
\end{tabularx}
\caption{Σύγκριση UDP και TCP}\label{table:udp-tcp}
\end{table}
\FloatBarrier

\subsection{Δομή πακέτου UDP}
Η δομή ενός πακέτου UDP περιγράφεται αναλυτικά στο \href{http://tools.ietf.org/html/rfc768 RFC 768}{πρότυπο RFC 768} όπως αναφέρθηκε παραπάνω.
Στον \hyperref[table:udp-header]{πίνακα \ref{table:udp-header}} φαίνεται η δομή ενός UDP πακέτου.

Κάθε πακέτο UDP έχει μία κεφαλίδα (header) που αναφέρει τα χαρακτηριστικά του. Η κεφαλίδα περιλαμβάνει 4 πεδία (λίγα σε σχέση με πρωτόκολλα όπως το TCP).
Δύο από τα τέσσερα πεδία είναι προαιρετικά
(\textcolor{blue!25}{χρωματισμένα} στον \hyperref[table:udp-header]{πίνακα \ref{table:udp-header}}).
Τα πεδία είναι τα εξής:
\begin{itemize}
\item Source port (προαιρετικό).
Η πόρτα του αποστολέα από την οποία προήλθε το πακέτο. Εάν ο παραλήπτης επιθυμεί να στείλει κάποια απάντηση, θα πρέπει να την στείλει στην πόρτα αυτήν.

\item Destination port.
Η πόρτα του παραλήπτη στην οποία θα πρέπει να παραδοθεί το πακέτο.

\item Length
Περιλαμβάνει το μέγεθος του πακέτου σε bytes. Το μικρότερο δυνατό είναι 8 bytes, αφού η κεφαλίδα αυτή καθ' αυτή καταλαμβάνει τόσο χώρο.
Θεωρητικά, το μέγεθος του UDP πακέτου δεν μπορεί να ξεπερνάει τα 65527 bytes.

\item Checksum (προαιρετικό).
Χρησιμοποιείται για επαλήθευση της ορθότητας του πακέτου στο σύνολό του, δηλαδή τόσο της κεφαλίδας όσο και των δεδομένων.
Υπολογίζεται μέσω του Standard Internet Checksum αλγορίθμου (RFC 1071)\cite{rfc1071}\cite{udp-erg-abdn}.
\end{itemize}

\begin{table}
\centering
\begin{tabular}{|c|c|c|}
\hline
{}&\bfseries{Bits 0-15}&\bfseries{Bits 16-31}\\\hline
0&\cellcolor{blue!25}Source Port&Destination Port\\\hline
32&Length&\cellcolor{blue!25}Checksum\\\hline
64&\multicolumn{2}{c|}{Data}\\\hline
\end{tabular}
\caption{UDP πακέτο}\label{table:udp-header}
\end{table}
