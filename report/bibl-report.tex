\chapter{Βιβλιογραφική αναφορά}
\section{Το πρωτόκολλο UDP}
Το πρωτόκολλο UDP (User Datagram Protocol ή Universal Datagram Protocol)
είναι ένα από τα βασικά πρωτόκολλα που χρησιμοποιούνται στο Internet.
Σχεδιάστηκε το 1980 και ορίζεται στο RFC 768\cite{rfc768}\cite{wiki:udp}.

Αποτελεί ένα απλό,
\href{https://en.wikipedia.org/wiki/Connectionless_communication}{connectionless}
πρωτόκολλο στο οποίο δεν υπάρχουν διάλογοι
\href{https://en.wikipedia.org/wiki/Handshaking}{"handshaking"}.
Δηλαδή, η επικοινωνία μέσω του UDP δεν προσφέρει την δυνατότητα στις συνδεδεμένες συσκευές να "θυμούνται" που βρίσκονται σε μια "συνομιλία" ανταλλαγής μηνυμάτων.
Η επικοινωνία γίνεται μέσω σύντομων μηνυμάτων (γνωστών και ως
\href{https://en.wikipedia.org/wiki/Datagram}{datagrams}
) από τον έναν υπολογιστή στον άλλον μέσα σε ένα δίκτυο υπολογιστών.

Ένα από τα κύρια χαρακτηριστικά του UDP είναι ότι δεν εγγυάται αξιόπιστη επικοινωνία.
Τα πακέτα UDP που αποστέλλονται από έναν υπολογιστή μπορεί να φτάσουν στον παραλήπτη με λάθος σειρά, διπλά ή να μην φτάσουν καθόλου ανάλογα με την τρέχουσα κατάσταση του δικτύου.
Αντιθέτως, το πρωτόκολλο
\href{https://en.wikipedia.org/wiki/Transmission_Control_Protocol}{TCP}
διαθέτει όλους τους απαραίτητους μηχανισμούς ελέγχου και επιβολής της αξιοπιστίας και μπορεί να εγγυηθεί την αξιόπιστη επικοινωνία μεταξύ των υπολογιστών.
Ωστόσο, λόγω της έλλειψης των μηχανισμών αυτών, το πρωτόκολλο UDP καθιστάται αρκετά πιο γρήγορο και αποτελεσματικό

\subsection{Χρήσεις του UDP}
%TODO: own book citations.
Βάση των χαρακτηριστικών του, το UDP σήμερα χρησιμοποιείται κυρίως ως εξής:
\begin{itemize}
\item Σε εφαρμογές audio \& video streaming.
Για τις εφαρμογές αυτές είναι πολύ σημαντικό τα πακέτα να παραδοθούν στον παραλήπτη σε σύντομο χρονικό διάστημα, ούτως ώστε να μην υπάρχει διακοπή στην ροή του ήχου ή της εικόνας.

Στο TCP η διόρθωση λαθών απαιτεί αναμετάδοση των πακέτων που προσθέτει καθυστερήσεις και αυξημένο bandwith.
Το streaming απαιτεί ένα πρωτόκολλο που μπορεί να αγνοεί λάθη στα δεδομένα
\cite{video-audio-streaming}.

Δεδομένου ότι το UDP δεν εγγυάται την παράδοση των πακέτων,
ο δέκτης πρέπει να βασίζεται σε ανώτερα επίπεδα (πχ RTP) για να ανιχνεύσει απώλεια πακέτων
\cite{911156}\cite{wiki:rtp}.

\item \href{https://en.wikipedia.org/wiki/Domain_Name_System}{Domain Name System (DNS)}.

\item \href{https://en.wikipedia.org/wiki/Dynamic_Host_Configuration_Protocol}{Dynamic Host Configuration Protocol (DHCP)}.

\item Διαδικτυακά παιχνίδια (MMORPGs).\cite{games}

\item \href{https://en.wikipedia.org/wiki/Voip}{Voice over IP (VoIP)}.
\end{itemize}
