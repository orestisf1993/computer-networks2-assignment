\section{Echo τυχαίας καθυστέρησης}
Σύμφωνα με τα απαιτούμενα της εργασίας, πραγματοποιήθηκε για χρονική διάρκεια 4 λεπτών καταμέτρηση του χρόνου απόκρισης του συστήματος σε milliseconds για κάθε πακέτο \texttt{echo}.
Ο κωδικός που χρησιμοποιήθηκε είναι ο \echorequestcode{}.

Αναφέρονται τα εξής στοιχεία:
\begin{itemize}
\item Στην διάρκεια των 4 αυτών λεπτών εστάλησαν \echoxsend{} πακέτα \texttt{echo} και ελήφθησαν \echoxreceive{} πακέτα.
\item Ο μέγιστος χρόνος καθυστέρησης απόκρισης ενός πακέτου ήταν \maxechoxms{} ενώ ο ελάχιστος \minechoxms{}.
\item Ο μέσος χρόνος καθυστέρησης ενός πακέτου ήταν \meanechoxms{}.
\item Η διασπορά ήταν \varechoxms{}, ενώ η τυπική απόκλιση ήταν \stdechoxms{}.
\end{itemize}

Το διάγραμμα χρόνου απόκρισης φαίνεται στο \imageref{\plotxresponsetime\plotxhist}.
Μπορούμε να καταλάβουμε ότι το είδος η του χρόνου καθυστέρησης που παρενέβαλε ο server μεταξύ των πακέτων echo ακολουθεί Γκαουσιανή (κανονική) κατανομή.
Τα διαγράμματα ρυθμαπόδοσης φαίνονται στα σχήματα
\ref{fig:\plotxlimA\plotxlimAh},
\ref{fig:\plotxlimB\plotxlimBh}
και \ref{fig:\plotxlimC\plotxlimCh}.

\dplothere{\plotxresponsetime}{\plotxhist}{\echorequestcode{}: Χρόνος απόκρισης}
\dplothere{\plotxlimA}{\plotxlimAh}{\echorequestcode{}: Ρυθμαπόδοση ανά 8 seconds}
\dplothere{\plotxlimB}{\plotxlimBh}{\echorequestcode{}: Ρυθμαπόδοση ανά 16 seconds}
\dplothere{\plotxlimC}{\plotxlimCh}{\echorequestcode{}: Ρυθμαπόδοση ανά 32 seconds}
\FloatBarrier
