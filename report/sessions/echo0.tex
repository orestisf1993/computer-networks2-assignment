\section{Echo χωρίς καθυστέρηση}
Σύμφωνα με τα απαιτούμενα της εργασίας, πραγματοποιήθηκε για χρονική διάρκεια 4 λεπτών καταμέτρηση του χρόνου απόκρισης του συστήματος σε milliseconds για κάθε πακέτο \texttt{echo}, με απενεργοποιημένη την καθυστέρηση που παρεμβάλει ο server.
Ο κωδικός που χρησιμοποιήθηκε είναι ο \echozcode{}.

Αναφέρονται τα εξής στοιχεία:
\begin{itemize}
\item Στην διάρκεια των 4 αυτών λεπτών εστάλησαν \echozsend{} πακέτα \texttt{echo} και ελήφθησαν \echozreceive{} πακέτα.
\item Ο μέγιστος χρόνος καθυστέρησης απόκρισης ενός πακέτου ήταν \maxechozms{} ενώ ο ελάχιστος \minechozms{}.
\item Ο μέσος χρόνος καθυστέρησης ενός πακέτου ήταν \meanechozms{}.
\item Η διασπορά ήταν \varechozms{}, ενώ η τυπική απόκλιση ήταν \stdechozms{}.
\end{itemize}

Το διάγραμμα χρόνου απόκρισης φαίνεται στο \imageref{\plotzresponsetime\plotzhist}.
Τα διαγράμματα ρυθμαπόδοσης φαίνονται στα σχήματα
\ref{fig:\plotzlimA\plotzlimAh},
\ref{fig:\plotzlimB\plotzlimBh}
και \ref{fig:\plotzlimC\plotzlimCh}.

\dplothere{\plotzresponsetime}{\plotzhist}{\echozcode{}: Χρόνος απόκρισης}
\dplothere{\plotzlimA}{\plotzlimAh}{\echozcode{}: Ρυθμαπόδοση ανά 8 seconds}
\dplothere{\plotzlimB}{\plotzlimBh}{\echozcode{}: Ρυθμαπόδοση ανά 16 seconds}
\dplothere{\plotzlimC}{\plotzlimCh}{\echozcode{}: Ρυθμαπόδοση ανά 32 seconds}
\FloatBarrier
