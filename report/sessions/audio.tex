%- wave: song & random (2)
%- dpcm διαφορες & δειγματα (2) + κατανομες (2) :10
%- aq διαφορες & δειγματα (2) + κατανομες (2) :23
%- 2 aq: ακολουθίες mu & beta (4) :5 + 23
\section{Αρχεία ήχου}
\subsection{Κυματομορφές}
    \subsubsection{Κομμάτι από την εικονική γεννήτρια συχνοτήτων}
    Με τον κωδικό \Rcode{} λάβαμε την κυματομορφή που φαίνεται στο
    \plotandref{\Rbuffer}{\Rcode{}: Κυματομορφή από την εικονική γεννήτρια συχνοτήτων}.
    Επίσης, παρουσιάζονται:
    \begin{itemize}
    \item \dplotandref{\Rdecoded}{\Rhist}{\Rcode{}: Αποκωδικοποιημένη κυματομορφή}:
    Η αποκωδικοποιημένη κυματομορφή.
    \item \dplotandref{\Rddecoded}{\Rdhist}{\Rcode{}: Διαφορές αποκωδικοποιημένης κυματομορφής}:
    Οι διαφορές της αποκωδικοποιημένης κυματομορφής.
    \item \dplotandref{\Rmean}{\Rstep}{\Rcode{}: Εξέλιξη $\mu$ και $\beta$}:
    Η εξέλιξη του μέσου όρου και του βήματος, σύμφωνα με τις πληροφορίες που λήφθηκαν από την Ιθάκη.
    \end{itemize}
    \FloatBarrier
    \subsubsection{Κομμάτι από το πειραματικό ρεπερτόριο μουσικής}
    Για το τραγούδι 10 (\href{https://www.youtube.com/watch?v=raK2GNAFhJY}{Tanto Project - Perfect Colour})
    λάβαμε την κυματομορφή που φαίνεται στο \plotandref{\Dbuffer}{\Dcode{}: Κυματομορφή από το πειραματικό ρεπερτόριο μουσικής}.
    Περισσότερες πληροφορίες στα \ref{sub:dpcm}, \ref{sub:aq} και \ref{sub:aq-m-b}.
\subsection{DPCM}\label{sub:dpcm}
    Για το τραγούδι 10 (\href{https://www.youtube.com/watch?v=raK2GNAFhJY}{Tanto Project - Perfect Colour}) το οποίο λήφθηκε μέσω του κωδικού \Dcode{} παρουσιάζονται:
    \begin{itemize}
    \item \dplotandref{\Ddecoded}{\Dhist}{\Dcode{}: Αποκωδικοποιημένη κυματομορφή}:
    Η αποκωδικοποιημένη κυματομορφή.
    \item \dplotandref{\Dddecoded}{\Ddhist}{\Dcode{}: Διαφορές αποκωδικοποιημένης κυματομορφής}:
    Οι διαφορές της αποκωδικοποιημένης κυματομορφής.
    \end{itemize}

    Παρατηρούμε ότι η κυματομορφή του κομματιού έχει μικρές διαφορές στην αρχή, άλλα μεγαλύτερες όταν αρχίζει το εντονότερο μέρος του τραγουδιού.
    Επίσης, η επαναληπτική φύση του τραγουδιού ("beatακι") μπορεί να παρατηρηθεί και στην περιοδικότητα της κυματομορφής.
\FloatBarrier
\subsection{AQ-DPCM}\label{sub:aq}
    Για το τραγούδι 23 (\href{https://www.youtube.com/watch?v=tuXOTlYOSEA}{Residence Deejays - Sexy Love}) το οποίο λήφθηκε μέσω του κωδικού \Acode{} παρουσιάζονται:
    \begin{itemize}
    \item \dplotandref{\Adecoded}{\Ahist}{\Acode{}: Αποκωδικοποιημένη κυματομορφή}:
    Η αποκωδικοποιημένη κυματομορφή.
    \item \dplotandref{\Addecoded}{\Adhist}{\Acode{}: Διαφορές αποκωδικοποιημένης κυματομορφής}:
    Οι διαφορές της αποκωδικοποιημένης κυματομορφής.
    \end{itemize}

    Παρατηρούμε ότι το ύψος της κυματομορφής είναι σχετικά σταθερό, καθώς το τραγούδι είναι ήδη σε αρκετά έντονο μέρος (αρχίζει από την μέση, όχι από την αρχή).
    Γενικά, δεν παρατηρείται κάποια περιοδικότητα.
    Αυτό μπορεί να οφείλεται στην ύπαρξη φωνητικών στο τραγούδι.
\subsection[AQ-DPCM: εξέλιξη \emph{μ} και \emph{β}]{AQ-DPCM: εξέλιξη $\mu$ και $\beta$}\label{sub:aq-m-b}
    \begin{itemize}
    \item Για το τραγούδι 5 (\href{https://www.youtube.com/watch?v=YPwtJ89jes4}{Kylie Minogue - Can't Get You Out Of My Head})
    το οποίο λήφθηκε μέσω του κωδικού \Acode{}
    η εξέλιξη των $\mu$ και $\beta$ φαίνεται στο
    \dplotandref{\Ameansecond}{\Astepsecond}{\Acode{}: Εξέλιξη $\mu$ και $\beta$}.
    \item Για το τραγούδι 23 (\href{https://www.youtube.com/watch?v=tuXOTlYOSEA}{Residence Deejays - Sexy Love})
    το οποίο λήφθηκε μέσω του κωδικού \Acodesecond{}
    η εξέλιξη των $\mu$ και $\beta$ φαίνεται στο
    \dplotandref{\Amean}{\Astep}{\Acodesecond{}: Εξέλιξη $\mu$ και $\beta$}.
    \end{itemize}

\FloatBarrier
