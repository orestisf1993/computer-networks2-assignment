\chapter{Η εφαρμογή (\appname{})}
\section{Script \scriptname{}}
Για την γρήγορη εξαγωγή των κωδικών από το site \url{http://ithaki.eng.auth.gr/netlab/index.html} γράφτηκε ένα script σε
\href{https://www.python.org/}{python3}
όπου πραγματοποιεί τη σύνδεση (login) με το site με τα σχετικά credentials και εξάγει σε ένα αρχείο τους κωδικούς σε μορφή
\href{https://en.wikipedia.org/wiki/JSON}{json} ώστε να μπορούν εύκολα να διαβαστούν οι τελευταίοι κωδικοί αυτόματα από την εφαρμογή \appname{}.
Χρησιμοποιήθηκαν οι εξωτερικές βιβλιοθήκες:
\begin{itemize}
\item \href{http://docs.python-requests.org/en/master/}{requests} για την πραγματοποίηση της συνεδρίας, και το downloading των σχετικών σελίδων.
\item \href{http://www.crummy.com/software/BeautifulSoup/}{BeautifulSoup} για το parsing των \texttt{.html} αρχείων για την εξαγωγή των κωδικών.
\end{itemize}

Καθώς δεν είναι δυνατό το ανέβασμα επιπλέον αρχείου με το όνομα \scriptname{} ο κώδικας παρατίθεται εδώ:
\begin{code}
\inputminted[frame=single, breaklines=true, linenos=true, python3=true]{python}{../extract-codes.py}
\caption{Το script \scriptname{}}
\label{listing:extract-codes}
\end{code}

