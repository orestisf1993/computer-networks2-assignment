\newcommand{\codeRef}[1]{\hyperref[section:#1]{\mintinline{java}!#1()!}}
\chapter{Η εφαρμογή (\appname{})}
\section{Script \scriptname{}}
Για την γρήγορη εξαγωγή των κωδικών από το site \url{http://ithaki.eng.auth.gr/netlab/index.html} γράφτηκε ένα script σε
\href{https://www.python.org/}{python3}
όπου πραγματοποιεί τη σύνδεση (login) με το site με τα σχετικά credentials και εξάγει σε ένα αρχείο τους κωδικούς σε μορφή
\href{https://en.wikipedia.org/wiki/JSON}{json} ώστε να μπορούν εύκολα να διαβαστούν οι τελευταίοι κωδικοί αυτόματα από την εφαρμογή \appname{}.
Χρησιμοποιήθηκαν οι εξωτερικές βιβλιοθήκες:
\begin{itemize}
\item \href{http://docs.python-requests.org/en/master/}{requests} για την πραγματοποίηση της συνεδρίας, και το downloading των σχετικών σελίδων.
\item \href{http://www.crummy.com/software/BeautifulSoup/}{BeautifulSoup} για το parsing των \texttt{.html} αρχείων για την εξαγωγή των κωδικών.
\end{itemize}

Καθώς δεν είναι δυνατό το ανέβασμα επιπλέον αρχείου με το όνομα \scriptname{} ο κώδικας παρατίθεται εδώ:
\begin{code}
\inputminted[frame=single, breaklines=true, linenos=true, python3=true]{python}{../extract-codes.py}
\caption{Το script \scriptname{}}
\label{listing:extract-codes}
\end{code}

\section{Χρήση βιβλιοθηκών \& imports}
\begin{code}
\begin{minted}{java}
import com.google.gson.Gson;
import com.google.gson.JsonObject;
import com.google.gson.stream.JsonReader;

import javax.sound.sampled.AudioFormat;
import javax.sound.sampled.AudioSystem;
import javax.sound.sampled.LineUnavailableException;
import javax.sound.sampled.SourceDataLine;
import java.io.*;
import java.net.*;
import java.nio.ByteBuffer;
import java.nio.ByteOrder;
import java.util.Arrays;
import java.util.logging.ConsoleHandler;
import java.util.logging.Handler;
import java.util.logging.Level;
import java.util.logging.Logger;

import static javax.xml.bind.DatatypeConverter.printHexBinary;
\end{minted}
\caption{Imports στο \appname}
\end{code}

Χρησιμοποιήθηκαν οι εξής βιβλιοθήκες:
\begin{itemize}
\item
\href{https://github.com/google/gson}{\mintinline{java}!com.google.gson.Gson!}:
\begin{displayquote}
Gson is a Java library that can be used to convert Java Objects into their JSON representation. It can also be used to convert a JSON string to an equivalent Java object.
\end{displayquote}
Βιβλιοθήκη της google, για την ανάγνωση και parsing αρχείων τύπου \texttt{JSON}.
Το σχετικό dependency στο maven είναι:
\begin{minted}{xml}
<dependencies>
    <dependency>
        <groupId>com.google.code.gson</groupId>
        <artifactId>gson</artifactId>
        <version>2.6.2</version>
    </dependency>
</dependencies>
\end{minted}
Χρησιμοποιείται αποκλειστικά στη μέθοδο \codeRef{initVariables}.

\item
\href{https://docs.oracle.com/javase/8/docs/api/javax/sound/sampled/package-summary.html}{\mintinline{java}!javax.sound.sampled!}:
\begin{displayquote}
Provides interfaces and classes for capture, processing, and playback of sampled audio data.
\end{displayquote}
Για την αναπαραγωγή ήχου που λαμβάνεται σε μορφή \mintinline{java}!byte[]! (array).
Χρησιμοποιείται στην μέθοδο \codeRef{playMusic}.

\item
\href{https://docs.oracle.com/javase/8/docs/api/java/io/package-summary.html}{\mintinline{java}!java.io.*!}:
\begin{displayquote}
Provides for system input and output through data streams, serialization and the file system.
\end{displayquote}
Χρησιμοποιείται για την διαχείριση αρχείων. Κυρίως για την αποθήκευση των αποτελεσμάτων.

\item
\href{https://docs.oracle.com/javase/8/docs/api/java/net/package-summary.html}{\mintinline{java}!java.net.*!}:
\begin{displayquote}
Provides the classes for implementing networking applications.
\end{displayquote}
Περιλαμβάνει κλάσεις διαχείρισης δικτυακών πόρων.
Παρέχει τις πολύ βασικές κλάσεις για την εφαρμογή μας:
\mintinline{java}!DatagramPacket! και \mintinline{java}!DatagramSocket!.
Χρησιμοποιείται σε όλες τις μεθόδους που έχουν σχέση με τον server της Ιθάκης.

\item
\href{https://docs.oracle.com/javase/8/docs/api/java/nio/package-summary.html}{\mintinline{java}!java.nio!}:
\begin{displayquote}
Defines buffers, which are containers for data, and provides an overview of the other NIO packages.
\end{displayquote}
Χρησιμοποιείται για διάφορες λειτουργικότητες σχετικές με buffers.

\item
\href{https://docs.oracle.com/javase/8/docs/api/java/util/package-summary.html}{\mintinline{java}!java.util!}:
\begin{displayquote}
Contains the collections framework, legacy collection classes, event model, date and time facilities, internationalization, and miscellaneous utility classes (a string tokenizer, a random-number generator, and a bit array).
\end{displayquote}
\sloppy Χρησιμοποιούνται διάφορες βασικές λειτουργίες όπως:
\mintinline{java}!java.util.logging! και \mintinline{java}!java.util.Arrays!.
\end{itemize}
